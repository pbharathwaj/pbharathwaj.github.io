\documentclass[11pt,a4paper,notitlepage]{article}
\usepackage{hyperref}
\usepackage{amsfonts}
\usepackage{amsmath}
\usepackage{tikz}
\usepackage{booktabs}
\usepackage{tabularx}
\usepackage{amssymb}

\newcommand{\OO}{\mathrm{O}}
\newcommand{\KK}{\mathrm{K}}
\newcommand{\Norm}{\mathrm{Norm}}



\begin{document}

\section{Introduction}

\subsection{Dedekind--Zeta function}
Let $\KK$ be a totally real number field. Let $\mathfrak{a}$ be an ideal inside the ring of integers $\OO_\KK$. Consider the Dedekind-Zeta function: 

\[\zeta_K(s) =  \sum \limits_{\mathfrak{b}}  \frac{1}{\Norm(\mathfrak{b})^s},\]
along with the associated partial Dedekind--Zeta functions:
\[\zeta_K(\mathfrak{a},s) =  \sum \limits_{\mathfrak{b} \sim \mathfrak{a}}  \frac{1}{\Norm(\mathfrak{b})^s},\]
The first sum is taken with respect to all the integral ideals $\mathfrak{b}$ in $\OO_\KK$. The second sum is taken with respect to all the integral ideals $\mathfrak{b}$ that are in the same class as $\mathfrak{a}$ in the narrow ray class group of $\KK$. 

\subsection{Motivating questions}
Here's an overarching and sequential list of motivating questions: 
\begin{itemize}
\item (\textbf{analytic continuation}) Do the zeta functions $\zeta_K(s)$ and $\zeta_K(\mathfrak{a},s)$  have an analytic (or meromorphic) continuation to the entire complex plane? 
\item (\textbf{rationality}) Having established analytic continuation, can we conclude that the zeta values \textit{at non-positive integers} are rational?
\item (\textbf{integrality}) Having established rationality, for what primes $p$ are these zeta-values $p$-integral? 
\item (\textbf{Kummer congruences}) Having established $p$-integrality, how do we construct \textit{$p$-adic $L$-functions} associated to these zeta values?
\end{itemize}
 
 \subsection{Rationality}
 The main motivating question for our study group for this semester will be the question of \textit{rationality} of these zeta values. This is known as the \textit{Siegel--Klingen} rationality theorem. There are three celebrated proofs of this result, attributed to:
 \begin{enumerate}
 \item Siegel--Klingen (\textit{constant terms of Eisenstein series}),
 \item Shintani, and
 \item Sczech (\textit{cohomological}).
 \end{enumerate}

Our goal will be to read the methods of \textbf{Shintani} and \textbf{Sczech}. As a reminder to our eventual goal of studying $p$-adic $L$-functions, it is worth noting that in this set up, the  constructions of the $p$-adic $L$-functions by 
\begin{enumerate}
\item Deligne--Ribet \cite{MR579702},
\item Cassou--Nogu\`es \cite{MR522119,MR524276}, and
\item Charollois--Dasgupta \cite{MR3272012}.
\end{enumerate}

rely on the methods of Siegel--Klingen, Shintani and Sczech respectively. 

\[\text{Siegel--Klingen} \rightarrowtail \text{Deligne--Ribet}\]
\[\text{Shintani} \rightarrowtail \text{Cassou--Nogu\`es}\]
\[\text{Sczech} \rightarrowtail \text{Charollois--Dasguptas}\]






\section{Plan for the study group: Shintani's method}
\textbf{References for Shintani's method}
\begin{enumerate}
\item Shintani's \href{shintani-evaluation.pdf}{article} \cite{MR427231}	
\item Hida's \href{hida-elementary.pdf}{book} 	\cite{MR1216135}
\end{enumerate}
The study group will meet on \textbf{Tuesdays, 10:30 am}.
\begin{center}
\begin{tabularx}{\textwidth}{p{5cm}p{5cm}p{5cm}p{5cm}}
\toprule  
Date & Speaker & Section  & Summary  \\ 
\midrule \\
18 Jan & Bharath & Introduction  & Discuss the case $K=\mathbb{Q}$. \\
25 Jan & Bharath & Theorem 1 in \cite{MR427231}  $\implies$ Siegel--Klingen rationality & Pages 393-394, 404 \\
1 Feb & Shaunak & Proposition 1 and Corollary to Proposition 1 in \cite{MR427231} & Pages 396-398 \\
8 Feb & Radhika & Lemma 2 and Corollary to Lemma 2 in \cite{MR427231}	& Pages 398-400 \\
15 Feb & Mihir  & Lemma 3 and Proposition 4	 in \cite{MR427231} & Pages 400-402 \\
22 Feb & Mahesh & Proposition 4 and Theorem 1 in  \cite{MR427231} & Pages 402-404 \\
\bottomrule \\
\end{tabularx}
\end{center}
	
\section{Plan for the study group: Sczech's method}	
	\textbf{References for Sczech's method}
\begin{enumerate}
\item \href{sczech-eisenstein.pdf}{Sczech's Inventiones  article}	 \cite{MR1231838}.
\item \href{sczech_real_quadratic_fields.pdf}{Sczech's Commentarii  article} \cite{MR1171300}.
\end{enumerate}
In our talks, we will have to take the $Q$-limit formula (Theorem 2 in \cite{MR1231838}, Theorem 1 in \cite{MR1171300}) for granted.
\begin{center}
\begin{tabularx}{\textwidth}{p{6cm}p{5cm}p{5cm}p{5cm}}
\toprule  
Date & Speaker & Section  & Summary  \\ 
\midrule \\
Mar 1 & Bharath & Introduction + Eisenstein cocycle  & Sections 2.2-2.4 in \cite{MR1171300}, Pages 368-377, Lemma 1, Theorems 2-4\\ \\
Mar 8 & Bharath &  Limit formula + Trignometric and Bernoulli cocycles  & Sections 2.2-2.4 in \cite{MR1171300}, Pages 368-377, Lemma 1, Theorems 2-4  \\ \\
Mar 15 & Radhika &  $L$-functions in real quadratic fields & Section 2.5 in \cite{MR1171300}, Pages 377-381, Theorems 5-6 \\ \\ 
 Mar 22  & Mahesh & 	Properties of the rational cocycle and the Eisenstein cocycle & Sections 2.1 and 2.2 in \cite{MR1231838}, Pages 585-592, Lemmas 1-4, Theorem 3-4 \\ \\
  Mar 29  & Mahesh & 	Properties of the rational cocycle and the Eisenstein cocycle & Sections 2.1 and 2.2 in \cite{MR1231838}, Pages 585-592, Lemmas 1-4, Theorem 3-4 \\ \\
Apr 5  & Shaunak & 	Special values of $L$-series in terms of Eisenstein cocycle & Sections 3.1 in \cite{MR1231838}, Pages 592-595  \\ \\
Apr 12  & Shaunak &  Proof of Lemma 6 &  Section 3.2 in \cite{MR1231838}, Pages 595-597 \\ \\
Apr 19  & Mihir &  A finite expression for the Eisenstein cocycle and rationality &  Section 4.1 in \cite{MR1231838}, Pages 597-601, Theorem 6, Lemma 7, and Corollaries \\ \\
\bottomrule \\
\end{tabularx}
\end{center}

More references:
\begin{enumerate}
\item \href{gunnells_sczech.pdf}{An article by Gunnells and Sczech} \cite{MR1980994}.
\item \href{Charollois-Sczech_EMS.pdf}{An article by Charollois and Sczech}\cite{MR3526308}. 
\item \href{0001011.pdf}{An article by Chinta, Gunnells and Sczech} \cite{MR1850609}.
\end{enumerate}

	\bibliographystyle{abbrv}
	\bibliography{references.bib}
\end{document}